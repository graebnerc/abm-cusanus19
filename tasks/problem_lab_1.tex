\documentclass[25pt,a4paper]{article}
\usepackage[pdftex,dvipsnames]{color}
\usepackage[german]{babel}
\usepackage[utf8]{inputenc}

\usepackage{geometry}
\usepackage[display]{texpower}
\usepackage{amsmath, graphicx,amsfonts}
%\usepackage[latin1]{inputenc}
\usepackage{color}
\usepackage{multicol,multirow}
\usepackage{multirow,colortbl,hhline}
\usepackage{epstopdf}
\usepackage{hyperref}
\usepackage{listings}

\definecolor{lgrey}{rgb}{0.9,0.9,0.9}

\geometry{headsep=3ex,hscale=0.9}
\parindent=0.2in
\textwidth=6.2in
\textheight=9.2in
\oddsidemargin=0.1in
\evensidemargin=0.2in
\headsep=0.2cm
\pagestyle{headings}

\emergencystretch=4cm

\begin{document}
\setlength{\emergencystretch}{0.5cm}
\pagestyle{empty}
\newcommand{\thegroup}{3}
\newcommand{\ifcasewrapper}[1]
{\ifcase#1\relax\or 
    \newcommand{\theothergroup}{2} %1
\or 
    \newcommand{\theothergroup}{1} %2
\or 
    \newcommand{\theothergroup}{4} %3
\or 
    \newcommand{\theothergroup}{3} %4
\fi
}
\ifcasewrapper{\thegroup}
\section*{Complexity Economics: Problem Set Lab 1}

\newcounter{counter1}

\begin{enumerate}
  \item \label{prob:2} Write a python script to compute the sum of all integer numbers between $0$ and $100$ that are not evenly (without remainder) divisible by either $4$ or $5$. That is, the numbers $$\left\{1, 2, 3, 6, 7, 9, 11, 13, 14, 17, 18, ...\right\}.$$ %3000
  \item Consider the code in problem \ref{prob:2} again and rewrite the computation as a function such that the intervals (from $0$ to $100$ in problem \ref{prob:2}) can be passed as arguments. Use this function to compute the sums of all integers not divisible by $4$ or $5$ in the following intervals $\left[100, 300\right]$, $\left[100, 300\right]$, $\left[10000, 20000\right]$. %24000, 2400000, 90000000  
  \item \label{prob:4} Consider the Fibonacci series, defined as 
  $$a_n = a_{n-1} + a_{n-2} \quad\quad\quad \textup{ for } n>2 \quad\quad \textup{ with } a_1=1, a_2=1,$$
  thus $\left\{1, 1, 2, 3, 5, 8, 13, 21, 34, 55\right\}$.
  Write a python function to compute the n'th Fibonacci number. E.g, the function called with argument $9$ should return $34$; with argument $10$, it should return $55$ etc.
  \item \label{prob:5} Use the function from problem \ref{prob:4} to compute the 40th Fibonacci number. %102334155 
%  \item \label{prob:6} 
\setcounter{counter1}{\value{enumi}}
\end{enumerate}

$$$$
$$$$
$$$$


\begin{enumerate}
\setcounter{enumi}{\value{counter1}}
  \item \label{prob:1} Consider the following code listings. In each code listing there is a mistake. Correct the mistakes.
\setcounter{counter1}{\value{enumi}}
\end{enumerate}

\small
\begin{lstlisting}[language=Python,frame=single,numbers=left,title=Script 0]
"""Sum integers from 1 to 10"""
result = 0
for i in range(10)
    result += i
\end{lstlisting}
\normalsize

\small
\begin{lstlisting}[language=Python,frame=single,numbers=left,title=Script 1]
"""Sum integers from 1 to 10"""
result = 0
i = 1
while i <= 10
    result += i
\end{lstlisting}
\normalsize

\newpage

\small
\begin{lstlisting}[language=Python,frame=single,numbers=left,title=Script 2]
"""Function to compute relation of two arguments a and b"""
def rel(a, b):
    return a / b

a = "2"
b = 20
result = rel(a, b)  
\end{lstlisting}
\normalsize

\small
\begin{lstlisting}[language=Python,frame=single,numbers=left,title=Script 3]
"""Function to compute factorial of integer b"""
def factorial(result, a):
    if a > 1:
        result = factorial(result, a-1)
    result *= a
    return result

b = 20
result = factorial(b)  
\end{lstlisting}
\normalsize

\small
\begin{lstlisting}[language=Python,frame=single,numbers=left,title=Script 4]
"""Function to compute factorial of integer b"""
def factorial(result, a):
    if a > 1:
        result = factorial(result, a-1)
    result *= a
  
b = 20
result = factorial(1, b)  
\end{lstlisting}
\normalsize

\small
\begin{lstlisting}[language=Python,frame=single,numbers=left,title=Script 5]
"""Function to compute factorial of integer b"""
def factorial(result, a):
    while a > 1:
        result = factorial(result, a-1)
    result *= a
    return result

b = 20
result = factorial(1, b)  
\end{lstlisting}
\normalsize



\end{document}
